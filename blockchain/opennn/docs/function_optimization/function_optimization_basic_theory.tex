\index{global minimum condition}
\index{minimal argument} 
\index{minimization}
\index{maximal argument}
\index{maximization} 
\index{global minimum}
\index{local minimum}
\index{unimodal function}
\index{multimodal function}
\index{local minimum condition}

The variational problem is formulated in terms of finding a function
which is an extremal argument of some performance functional. On the
other hand, the function optimization problem is formulated in terms
of finding a vector which is an extremal argument of some performance
function.

While neural networks naturally leads to the solution of
variational problems, \texttt{OpenNN} provides a workaround for function
optimization problems by means of the independent parameters. 

Function optimization refers to the study of problems in which the
aim is to minimize or maximize a real function. In this way, the
performance function defines the optimization problem itself.

The formulation of a function optimization problem requires:

\begin{itemize}
\item[-] A neural network.
\item[-] A performance functional.
\item[-] A training strategy.
\end{itemize}

\subsection*{Neural network}
\index{unconstrained function optimization problem}
\index{number of variables}
\index{domain, objective function}
\index{image, objective function}

The independent parameters of a neural network spans a vector space to represent the possible solutions of a function optimization problem.

\subsection*{Performance functional}

The function to be optimized is called the performance function. The
domain of the objective function for a function optimization problem
is the set of independent parameteres, and the image of that function
is the set of real numbers. The number of variables in the objective function
is the number of independent parameters. 

A function optimization problem can be specified by a set of
constraints, which are equalities or inequalities that the solution
must satisfy. Such constraints are expressed as functions. 

Thus, the
constrained function optimization problem can be formulated so as to find a vector 
 such that the constrainst functions are zero 
and for which the performance function takes on a minimum value.

In other words, the constrained function optimization problem
consists of finding an argument which makes all the constraints to
be satisfied and the objective function to be an extremum. The
integer $l$ is known as the number of constraints in the function
optimization problem.

\subsection*{Training strategy}

The training strategy is the solving strategy
 for the optimization problem. 
If possible, the quasi-Newton method should be applied here. 
If that fails, the evolutionary algorithm can be used. 

