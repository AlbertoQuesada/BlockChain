Neural networks have found a wide range of applications,
which include function regression, pattern recognition, time series
prediction, optimal control, optimal shape
design or inverse problems. A neural network can learn either from data sets or from mathematical models. 

\texttt{OpenNN} is a comprehensive class library which implements neural networks
 in the C++ programming language. 
This software tool can be used
for the whole range of applications mentioned above. \texttt{OpenNN}
also provides a workaround for the solution of function optimization
problems. The library has been released as the open source GNU
Lesser General Public License.

This manual is organized as follows: 
Chapter \ref{Preliminaries} provides some guidelines for installing the software and using some basic data structures, 
such as vectors and matrices. 
In Chapter \ref{NeuralNetworksBasis}, a brief introduction to the principal concepts of neural networks is given. 
Also, in Chapter \ref{SoftwareModelBasis} the most general software model of \texttt{OpenNN} is presented.
Chapters \ref{NeuralNetwork}, \ref{PerformanceFunctional} and \ref{TrainingStrategy} state the learning problem for neural networks and provide a collection of related algorithms. 
In Chapters \ref{FunctionRegression}, \ref{PatternRecognition}, \ref{OptimalControl}, \ref{OptimalShapeDesign} and \ref{InverseProblems} the most important learning tasks for neural networks are formulated and several practical applications are also presented. 
Finally, Chapter \ref{FunctionOptimization} explains how to solve function optimization problems by means of \texttt{OpenNN}.

%Appendixes \ref{SoftwareModel} and \ref{UnitTesting} present some activities related to the software engineering process of \texttt{OpenNN}. 
%Finally, Appendix \ref{NumericalIntegration} introduces some numerical integration algorithms. 
